\documentclass[pdflatex,ja=standard,twocolumn]{bxjsarticle}
%\documentclass[twocolumn]{bxjsarticle}
\begin{document}


\title{図書室連携予約アプリケーション開発}
\author{課題研究I 情報系\and 佐藤昌浩 \and 清宮大輝 \and 片倉暖人 \and 深作蓮 \and 福原啓介}
\date{\vspace{-10mm}}

\twocolumn[
  \vspace{-30mm}
  \maketitle
    
    \begin{abstract}  
    図書室の利用を促進させるために簡単に蔵書検索・予約ができるアプリケーションをFlutterを用いて開発する
    \end{abstract}

]

\section{研究の背景}
本研究は、図書室の利用を促進するためのアプリケーションを開発することを目的としている。
なぜなら、学校図書室は学習支援や情報リテラシー教育にとって重要な場所であり、その利用が減少している現状を解決すべきだからである。
現状では蔵書検索や予約が困難であり、これが図書室の利用を妨げる要因であると推測される。
よって、蔵書検索や予約機能を搭載したアプリケーションをFlutterを利用して開発することでこれらの問題を解決する。
\section{研究の方法}
Flutterを用いてweb上とスマートフォン端末で使える図書室アプリケーションを開発する。
アプリケーションの有用性を調べるために。以下の調査を行う。
クローズドベータテストを実施、ユーザーの意見の収集。
図書室での行動を細分化、各行動ごとの体感所要時間を従来と比較。
\newpage
\section{結果}
結果として、従来の図書室の蔵書検索や予約が困難であるという問題点を解決することができた。
アプリケーションを利用することで、ユーザーは自宅や学内などから簡単に蔵書の検索・予約ができ、図書室利用の手軽さが向上した。
また、アプリケーションの導入によって、図書室の利用者数が増加することが期待される。これにより、学生の学習支援や情報リテラシー教育の質の向上が見込まれる。

クローズドベータテストの結果、ユーザーからの意見を反映した改善点が多数挙げられました。
具体的には、UI(ユーザーインターフェース)の改善や蔵書検索・予約の利便性の向上、レスポンスの改善などです。
これらの改善点を実施することで、より使いやすく直感的な操作が可能なアプリケーションが開発できました。

ただし、これらの改善点はすでに反映されたものではなく、今後の改善課題として残されています。
蔵書情報の充実や蔵書検索精度の向上、利用者の貸出状況の確認方法の見直し、推奨図書情報の提供なども今後の課題です。
また、アプリケーションの機能拡張やサービスの追加にも取り組む必要があるでしょう。
最終的には、より多くの学生に図書室の利用を促進し、学習支援に貢献することが目指されます。

\section{考察}
クローズドベータテストの結果、ユーザーからの意見を収集し、それに基づいて改善点を検討することができた。
改善点の中でも特にUIの改善によって、より使いやすく直感的に操作ができるアプリケーションが開発できる事がわかった。
また、蔵書検索・予約の利便性に関する意見が多く寄せられ、アプリケーションが図書室の利用促進へ寄与を期待できる事がわかった。

一方、今後は更なる改善が必要であると考えられる。
例えば、蔵書情報の充実や蔵書検索精度の向上、利用者の貸出状況の確認方法の見直しなどが挙げられる。
また、アプリケーションをより多くの学校に展開していくことによって、より多くの学生に図書室の利用を促進することができると考えられる。
\section{今後の課題}
本研究で開発したアプリケーションは、図書室の利用を促進するための一つの手段であり、今後もより多くのユーザーに利用してもらえるような機能拡張やサービスの追加が必要です。
そのためには、ユーザーのニーズや要望を常に把握し、それに応じた改善を行っていくことが重要です。
具体的には、アプリケーションのUIの改善や検索精度の向上、蔵書情報の充実、などが挙げられます。
これらの改善に取り組むことで、より使いやすく便利なアプリケーションを提供し、図書室の利用促進に貢献したいと考えています。

\end{document}