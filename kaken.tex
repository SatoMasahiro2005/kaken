\documentclass[pdflatex,ja=standard,twocolumn]{bxjsarticle}
%\documentclass[twocolumn]{bxjsarticle}
\begin{document}


\title{図書室連携予約アプリケーション開発}
\author{課題研究I 情報系\and 佐藤昌浩 \and 清宮大輝 \and 片倉暖人 \and 深作蓮 \and 福原啓介}
\date{\vspace{-10mm}}

\twocolumn[
  \vspace{-30mm}
  \maketitle
    
    \begin{abstract}  
    図書室の利用を促進させるためにFlutterを用いて簡単に蔵書検索・予約ができるアプリケーションを開発する
    \end{abstract}

]

\section{研究の背景}
本研究は、図書室の利用を促進するためのアプリケーションを開発することを目的としている。
学校図書室は学習支援や情報リテラシー教育にとって重要な場所であり、その利用が減少している現状が問題視されている。
蔵書検索や予約が困難であることが挙げられるが、Flutterを利用したアプリケーションの開発に
よってこれらの問題を解決することが期待される。本研究は、Flutterを利用した学校図書室ア
プリケーションの開発を行い、学校図書室の利用促進に貢献することを目指す。
\section{研究の方法}
Flutterを用いてweb上とスマートフォン端末で使えるアプリケーションを開発する。
アプリケーションの有用性を調べるために。以下の調査を行う。
\begin{itemize}
  \item クローズドベータテストを実施、ユーザーの意見の収集
  \item 図書室での行動を細分化、各行動ごとの体感所要時間を従来と比較
\end{itemize}
\section{結果}
結果として、従来の図書室の蔵書検索や予約が困難であるという問題点を解決することができた。
アプリケーションを利用することで、利用者は自宅や学内などから簡単に蔵書の検索・予約ができ、図書室利用の手軽さが向上した。
また、アプリケーションの導入によって、図書室の利用者数が増加することが期待される。これにより、学生の学習支援や情報リテラシー教育の質の向上が見込まれる。

また、クローズドベータテストの結果、ユーザーからの意見を反映した改善点が多数あげられた。
具体的には、UIの改善や蔵書検索・予約の利便性の向上、レスポンスの改善、推奨図書情報の提供などが挙げられる。
これらの改善点を実施することにより、より使いやすく、直感的な操作が可能なアプリケーションが開発できた。

ただし、今後も改善の余地は残されており、蔵書情報の充実や蔵書検索精度の向上、利用者の貸出状況の確認方法の見直し、推奨図書情報の提供などが必要である。
また、アプリケーションの機能拡張やサービスの追加にも取り組む必要がある。最終的には、より多くの学生に図書室の利用を促進し、学習支援に貢献することが目指される。
\begin{itemize}
 \item クローズベータテストを実施したところ、ユーザーからの意見としては以下のようなものがあげられた
 \item 
\end{itemize}
 
\section{考察}
クローズドベータテストの結果、ユーザーからの意見を収集し、それに基づいて改善点を検討することができた。
特に、UI(ユーザーインターフェース)の改善によって、より使いやすく直感的に操作ができるアプリケーションが開発できた。
また、蔵書検索・予約の利便性に関する意見が多く寄せられ、アプリケーションが図書室の利用促進に寄与することが期待できることが分かった。

一方、今後は更なる改善が必要であると考えられる。
例えば、蔵書情報の充実や蔵書検索精度の向上、利用者の貸出状況の確認方法の見直し、推奨図書情報の提供などが挙げられる。
また、アプリケーションをより多くの学校に展開していくことによって、より多くの学生に図書室の利用を促進することができると考えられる。
\section{今後の課題}
本研究で開発したアプリケーションについては、まだ改善の余地があるため、今後も改善・アップデートを継続的に行っていく必要がある。
また、学校の教員や図書館員などとの協力体制を構築し、アプリケーションの利用促進に取り組む必要がある。
更に、利用者のニーズに応えるため、アプリケーションの機能拡張やサービスの追加を行い、より多くのユーザーに利用してもらえるようにすることが重要である。

\end{document}