\documentclass[pdflatex,ja=standard,twocolumn]{bxjsarticle}
%\documentclass[twocolumn]{bxjsarticle}
\begin{document}


\title{図書室連携予約アプリケーション開発}
\author{課題研究I 情報系\and 佐藤昌浩 \and 清宮大輝 \and 片倉はると \and 深作蓮 \and 福原啓介}
\date{\vspace{-10mm}}

\twocolumn[
  \vspace{-30mm}
  \maketitle
    
    \begin{abstract}  
    図書室の利用を促進させるためにFlutterを用いて簡単に蔵書検索・予約ができるアプリケーションを開発する
    \end{abstract}

]

\section{研究の背景}
本研究は、図書室の利用を促進するためのアプリケーションを開発することを目的としている。
学校図書室は学習支援や情報リテラシー教育にとって重要な場所であり、その利用が減少している現状が問題視されている。
蔵書検索や予約が困難であることが挙げられるが、Flutterを利用したアプリケーションの開発に
よってこれらの問題を解決することが期待される。本研究は、Flutterを利用した学校図書室ア
プリケーションの開発を行い、学校図書室の利用促進に貢献することを目指す。
\section{研究の方法}
Flutterを用いてweb上とスマートフォン端末で使えるアプリケーションを開発する。
アプリケーションの有用性を調べるために。以下の調査を行う。
\begin{itemize}
  \item クローズドベータテストを実施、ユーザーの意見の収集
  \item 図書室での行動を細分化、各行動ごとの体感所要時間を従来と比較
\end{itemize}
\section{結果}
班内の演習では、
\section{考察}
\section{今後の課題}


\end{document}